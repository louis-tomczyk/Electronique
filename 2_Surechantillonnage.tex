%%%%%%%%%%%%%%%%%%%%%%%%%%%%%%%%%%%%%%%%%%%%%%%%%%%%%%%%%%%%%%%%%%%%%%%%%%%%%%%%%%%%%%
% author                : louis tomczyk
% date of production    : 2024-02-11
% licence               : cc-by-nc-sa
%                         Attribution - Non-Commercial - Share Alike 4.0 International
%%%%%%%%%%%%%%%%%%%%%%%%%%%%%%%%%%%%%%%%%%%%%%%%%%%%%%%%%%%%%%%%%%%%%%%%%%%%%%%%%%%%%%


%\begin{comment}
%%%%%%%%%%%%%%%%%%%%%%%%%%%%%%%%%%%%%%%%%%%%%%%%%%%%%%%%%%%%%%%%%%%%%%%%%%%%
%%%%%%%%%%%%%%%%%%%%%%%%%%%%%%%%%%%%%%%%%%%%%%%%%%%%%%%%%%%%%%%%%%%%%%%%%%%%
%%%%%%%%%%%%%%%%%%%%%%%%%%%%%%%%%%%%%%%%%%%%%%%%%%%%%%%%%%%%%%%%%%%%%%%%%%%%
\newpage
\section{Impact sur le bruit}


%%%%%%%%%%%%%%%%%%%%%%%%%%%%%%%%%%%%%%%%%%%%%%%%%%%%%%%%%%%%%%%%%%%%%%%%%%%%
\subsection{Densité spectrale du bruit}

\bdefbox
    La variance du bruit est liée à l'erreur de quantification selon:
    \be
        \var{e} = \dfrac{1}{q}\d \int_{-q/2}^{q/2}e^2de = \dfrac{q^2}{12}
    \ee

    La Densité Spectrale de Puissance (DSP) du bruit lie la puissance du bruit
    à la plage de fréquence considérée selon:
    \be
        P_e = \int_{-f_e/2}^{f_e/2}~\gamma(f)df \tq \gamma\equiv [W/Hz]
    \ee

    où $f_e~[Hz]$ est la fréquence d'échantillonnage.
\edefbox

Or la puissance d'un signal aléatoire \underline{centré} est exactement sa variance:
\be
    \coloredbox{green!50!black}{P_e = \braket{|e|^2} = \var{e}}
\ee

On peut donc en déduire la DSP:
\be
    \gamma(f) = \dfrac{\var{e}}{f_e} = \dfrac{q^2}{12\d f_e}
\ee

\paragraph{Remarque}
Ce qui n'est rien d'autre que la variance d'une variable aléatoire uniforme
sur le segment $[-1,1]\d q/2$.

\bdbox
    Pour un signal purement sinusoïdal:
    \be
        x(t) = A\d cos(\omega t) \implies
        P_x = \dfrac{1}{T}\d\int_T|x^2(t)|dt = \dfrac{A^2}{2}
    \ee

    Le Rapport Signal à Bruit (RSB) s'écrit alors:
    \be
    \left\{
        \begin{array}{ccc}
        RSB &=& \dfrac{P_x}{P_e} = 6\d \dfrac{A^2}{q^2}\\
        q   &=& \dfrac{PE}{2^{N_b}}
        \end{array}
    \right.\implies
    RSB = \dfrac{3}{2}\d 2^{2N_b}\d \l(\dfrac{2A}{PE}\r)^2
    \ee

    Ou encore en échelle log:
    \be
        \coloredbox{green!50!black}{RSB_{dB} = 10\d \log_{10}(RSB)
        \approx 1.76+6.02\d N_b+20\log_{10}\l(\dfrac{2A}{PE}\r)}
    \ee

    \textbf{On remarque donc que pour une entrée sinusoïdale, chaque bit de quantification
    supplémentaire réduit par 4 le niveau de bruit ($6~[dB]$)!}
\edbox
%%%%%%%%%%%%%%%%%%%%%%%%%%%%%%%%%%%%%%%%%%%%%%%%%%%%%%%%%%%%%%%%%%%%%%%%%%%%
\subsection{Étalement du bruit}

Dans le cas d'un bruit blanc dans la bande spectrale considérée $B$
nous avons tout simplement la puissance du bruit $P_b$ qui s'écrit
en fonction de la densité spectrale de bruit $f\mapsto \gamma(f)$
qui est constante à $\gamma$:
\be
    P_b = B\d \gamma_0
\ee

Sur-échantillonner n'ajoute dans l'idéal pas de bruit car on ne change
pas les statistiques du signal que l'on échantillonne.
À puissance contante, si le support s'élargit (devient $B'$) sans modifier
la nature de $\gamma$ alors dans ce cas on a une nouvelle densité de bruit
$\gamma'$, illustrée sur la \autoref{fig:EtalementSurechantillonage} telle que:
\be
    \coloredbox{green!50!black}{P_b = B'\d \gamma' \tq B' = c_s\d B \tq c_s\geq 2}
\ee

où l'on introduit le coefficient de Shannon $c_s$.
On peut alors calculer $\gamma'$ selon:
\be
    P_b= P_b\iff \gamma' = \gamma_0/c_s
\ee

\input{Figures/Étalement_Spectre_Surechantillonage}
%%%%%%%%%%%%%%%%%%%%%%%%%%%%%%%%%%%%%%%%%%%%%%%%%%%%%%%%%%%%%%%%%%%%%%%%%%%%
\subsection{Taux de rejet}

Dans la bande d'intérêt $B$ la puissance de bruit $P_b^{(B)}$ a diminué:
\be
    \boxed{P_b^{(B)} = \gamma'\d B = P_b/c_s}
\ee

Et la puissance de bruit transférée dans la bande externe $P_b^{(BB')}$ est:
\be
    P_b^{(BB')} = P_b-P_b^{(B)} = \l(1-\frac{1}{c_s}\r)\d P_b
\ee

On peut alors définir un taux de rejet $t_{BB'}$ comme étant le rapport de la
puissance de bruit rejetée hors de la bande d'intérêt sur la puissance de
bruit initiale:
\be
    \boxed{t_{BB'} = \frac{P_b^{(BB')}}{P_b} = 1-\frac{1}{c_s}}
\ee

%%%%%%%%%%%%%%%%%%%%%%%%%%%%%%%%%%%%%%%%%%%%%%%%%%%%%%%%%%%%%%%%%%%%%%%%%%%%
\begin{comment}
\subsection{Bruits en télécommunications}

En télécommunications les signaux sont modulés et occupent une certaine bande
spectrale que l'on note $B_0$.
Dans le cas mono-canal avec un filtre de mise en forme en racine de cosinus-
surélevé, la largeur fréquentielle du signal est proportionnelle au débit de
symboles.
Or tout signal comporte du bruit provenant de sa génération.
Le bruit dans la bande spectrale intrinsèque au signal est a différencier du
bruit en dehors, pouvant provenir d'autres sources/phénomènes.

On peut alors distinguer le Rapport Signal à Bruit Propre (RSBP) au signal du
Rapport Signal à Bruit (RSB) prenant le tout.
On a alors naturellement:
\be
    RSBP>RSB.
\ee
Toute acquisition numérique implique donc un échantillonnage en respectant le
critère de Nyquist-Shannon s'accompagnant, comme mentionné précédemment, d'une
diminution du niveau de bruit dans la bande propre du signal.
Plus formellement, on note RSBP$_0$ le RSBP lorsqu'il n'y a pas de sur-échan
tillonage et RSBP quand il y en a.
\be
    RSBP_0 = P_s/(\gamma_0\d B_0) \And RSBP = P_s/(\gamma\d B_0)
\ee

On a alors:
\be
    RSBP = c_s\d RSBP_0
\ee
\end{comment}