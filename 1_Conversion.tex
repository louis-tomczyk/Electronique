%%%%%%%%%%%%%%%%%%%%%%%%%%%%%%%%%%%%%%%%%%%%%%%%%%%%%%%%%%%%%%%%%%%%%%%%%%%%%%%%%%%%%%
% author                : louis tomczyk
% date of production    : 2024-02-11
% licence               : cc-by-nc-sa
%                         Attribution - Non-Commercial - Share Alike 4.0 International
%%%%%%%%%%%%%%%%%%%%%%%%%%%%%%%%%%%%%%%%%%%%%%%%%%%%%%%%%%%%%%%%%%%%%%%%%%%%%%%%%%%%%%

%%%%%%%%%%%%%%%%%%%%%%%%%%%%%%%%%%%%%%%%%%%%%%%%%%%%%%%%%%%%%%%%%%%%%%%%%%%%
%%%%%%%%%%%%%%%%%%%%%%%%%%%%%%%%%%%%%%%%%%%%%%%%%%%%%%%%%%%%%%%%%%%%%%%%%%%%
%%%%%%%%%%%%%%%%%%%%%%%%%%%%%%%%%%%%%%%%%%%%%%%%%%%%%%%%%%%%%%%%%%%%%%%%%%%%
\section{Conversion d'une valeur analogique}

\bdefbox
	En entrée on a un signal à temps continu $(x)$ que l'on transforme en signal discret
	$(b_k)_{k\in J\subset \N}$.
	L'équation de définition du convertisseur est:
	\be
		x[k] = V_{min}+\l(\sum_{j=1}^{N_b+1} \dfrac{b_j[k]}{2^j}\r)\d PE+e
	\ee

	où:
	\bitem
		\item $PE=V_{max}-V_{min}$: la tension de Plein Échelle
		\item $V_{min/max}$: la tension min/max supportées par le CAN
		\item $e$ l'erreur de quantification
		\item $N_b$: le nombre de bits utilisés par la quantification
	\eitem

	On peu poser le pas de quantification:
	\be
		\coloredbox{green!50!black}{q = PE/2^{N_b}}
	\ee
\edefbox

Ainsi on peut réécrire la somme précédente selon:
\be
	\l(\sum_{j=1}^{N_b+1} \dfrac{b_j[k]}{2^j}\r)\d PE &=& 
	\sum_{j=1}^{N_b+1} \dfrac{b_j[k]}{2^j} \d PE+\dfrac{PE}{2^{N_b+1}}\\
	&=& \sum_{j=1}^{N_b+1} b_j[k]2^{N_b-j}\dfrac{PE}{2^{N_b}}+\dfrac{q}{2}\\
	&=& Nq+\dfrac{q}{2}
\ee


Ce qui permet de réécrire le signal $x$ à l'échantillon $k$ selon:
\be
	x[k] = V_{min}+\l(N+\dfrac{1}{2}\r)\d q+e
\ee

où la sortie numérique du convertisseur $N$ est:
\be
	\coloredbox{green!50!black}{N = \sum_{j=1}^{N_b+1} b_j[k]2^{N_b-j}}
\ee

\paragraph{Remarques}
On note:
\bitem
	\item $b_1$: le Most Significant Bit (MSB)
	\item $b_{N_b}$: le Least Significant Bit (LSB)
	\item $+q/2$: ce terme permet de centrer l'erreur: $e\in[-1,1]\d q/2$
\eitem

\bdbox
	Par exemple si on a:
	\be
		\left\{
			\begin{array}{ccc}
				PE & = & 2~[V]\tq V_{max} = -V_{m} = 1~[V]\\
				N_b &=& 3
			\end{array}
		\right.\implies q= 2/(2^3) = 1/4
	\ee

	Si $x[k] = -0.41~[V]$ alors:
	\be
			-0.41 = -1 + (N+0.5)/4+e \iff
			N = \dfrac{x[k]-e-V_{min}}{q}-\dfrac{1}{2} = 1.86-4\d e
	\ee

	Or l'erreur est encadrée:
	\be
		-\dfrac{q}{2} 	& <e< 		& \dfrac{q}{2}\\
		1.86+2q 		& >1.86-4e> & 1.86-2q\\
		2.36 			& >1.86-4e>	& 1.36
	\ee

	D'où $N=2$.

	Le mot binaire s'écrit donc ici:
	\be
		N = 2 	& = & \sum_{j=1}^{N_b+1} b_j[k]2^{3-j}\\
				& = & b_1[k]\d 2^2 + b_2[k]\d 2^1 + b_3[k]\d 2^0
	\ee

	Or $\forall j, b_j\in\{0,1\}$, donc nécessairement:
	\be
		x[k] = -0.41 \equiv (0~1~0)
	\ee

	L'erreur de quantification quant à elle:
	\be
		x[k] &=& V_{min}+b_1[k]\dfrac{PE}{2^1}+b_2[k]\dfrac{PE}{2^2}+b_3[k]\dfrac{PE}{2^3}+\dfrac{PE}{2^4}+e\\
			&=& -1+0+1\d 0.5+0+0.125+0+e \iff e = -0.035\\
			&=& -1+\l(2+\dfrac{1}{2}\r)\d \dfrac{1}{4}+e \iff e = -0.035
	\ee
\edbox
